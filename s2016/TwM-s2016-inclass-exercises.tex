% To teach: curation, presentation, feedback

% test account: labsmbatest, wharton1
% See {\it Notebook} 2008-6-4 pages 6--7,  2008-7-15 pages 38--40 and {\it Daybook}2008-8-14, pages 47--8.
% On the cosine measure of similarity:
% http://www.miislita.com/information-retrieval-tutorial/cosine-similarity-tutorial.html


 \def\year{2016}
 \def\lastclass{Wednesday, April 29, \year}
%  \def\matlabcasedue{9 p.m. on Monday, March 30, 2015}
%  \def\groupassignmentdue{5 p.m. on Tuesday, May 5, 2015}
%  \def\pythoncasedue{5 p.m. on Thursday, May 7, 2015}
% \def\nis{../../../books/RobustDecisionMaking/NewImages/}
% \def\mlb{MATLAB}
% \def\mb{{\it DAMbook}}
\def\noop#1{}
\def\place{TBA}
\def\figtop{\rule{\textwidth}{0.5mm}}
\def\figbot{\rule{\textwidth}{0.5mm}}
\noop{
\section{Readings}
\section{Lecture notes}
\section{Exercises}
\section{In class assignments}
\section{Case assignments}
  }
 
  
\newtheorem{exercise}{Exercise}
\documentclass[11pt]{book}
\makeindex
\usepackage{url}
\usepackage{palatino}
%
\usepackage{hyperref}
\usepackage{color}
\usepackage{amssymb}

\usepackage{makeidx}
%\usepackage{geometry}
%\geometry{textwidth=6.5in}

\usepackage{rotating} % for sideways and sidewaystable, etc. environments
% the following package, when present, lets the figures and tables orient oppositely on even and odd pages
\usepackage{lscape} 
\usepackage{natbib}

\makeindex
\newcount\draft
\draft=1
\newcount\instructor
\instructor=1
\newcount\answers
\answers=1
%%%%%%%%%%%%%%%%

\begin{document}

%\index{BMI!|see{body mass index}}

\pagenumbering{roman}
\setcounter{secnumdepth}{5}
\setcounter{tocdepth}{2}
\frontmatter
%\nocite{*}
%
\pagestyle{empty}
\centerline{\Huge OID 325: Thinking with Models}
\vskip 12 pt
\centerline{\Large Spring \year, \place}
\vskip 30 pt
\centerline{\Large In-Class Exercises}
\vskip 35 pt
\centerline{\Large Steven O. Kimbrough}
\vskip 35 pt
\centerline{Draft: \copyright\ \today}

\ifnum\draft=1
\vfill
{\footnotesize
\noindent\verb+$Id: TwM-s2016-inclass-exercises.tex 4952 2015-09-13 23:53:32Z sok $+
}
\fi
% \newpage

% \noindent Steven O. Kimbrough, %103 Bentley Avenue, Bala Cynwyd, PA 19004--2805. 
% Tel: (215) 898-5133.  Fax: (215) 898-3664. Email: kimbrough@wharton.upenn.edu. Web: %{\tt http://opim.wharton.upenn.edu/\symbol{126}sok/}.
% \url{http://opim.wharton.upenn.edu/~sok/}

\newpage
\pagestyle{plain}
\tableofcontents

%
\listoffigures

%
\listoftables

%\chapter{Executive Summary}
\chapter{Preface}

\mainmatter
\pagenumbering{arabic}
%
\pagestyle{headings}

%%%%%%%% Class 1, Introduction  %%%%%%%%%%%

%\input chapters/introduction.tex
\chapter{Introduction}

Today's exercise is simply a warm up for what is to come. With neighbors, form groups of 3--4 for today's exercise. Discuss among yourselves your modeling experiences and write down a short discussion and description of 2--4 models you are acquainted with. These could be models you have worked with or been instructed on, or they could be models you are interested in, etc. As a group, record your comments briefly in an electronic document (Word, \LaTeX , plain text, whatever) and upload it to Canvas, submitting as today's assignment. Be sure to put everyone's name on the document. If you can, upload a separate copy for each person in your group.

\chapter{Working with Patches}

Create a NetLogo model file called {\it Class2.nlogo}, put your work in it, and submit it for today's in-class exercises assignment.

\section{Basic}

\begin{enumerate}
\item Write a command procedure called \texttt{basic1} that when executed calls \texttt{clear-all} and then sets the color of all patches whose x-coordinate equals 6 or more to yellow. 

Here's a stub for you to work with:
\begin{verbatim}
to basic1
  clear-all
  [** Your code here. **]
end
\end{verbatim}


\ifnum\answers=1
\vskip 6 pt
\hrule
\vskip 3 pt
{\bf Answer: }
\begin{verbatim}
to basic1
  clear-all
  ask patches with [pxcor > 5] [set pcolor yellow]
end
\end{verbatim}
\vskip 3 pt
\hrule
\fi

\item Write a command procedure called \texttt{basic2} that when executed calls \texttt{clear-all} and then colors each patch yellow with probability 12/100, and then for each patch if it has 3 or more yellow neighbors, sets the patch color to red and sets it label to \texttt{***}

Here's a stub for you to work with:
\begin{verbatim}
to basic2
  clear-all
  [** Your code here. **]
end
\end{verbatim}

\ifnum\answers=1
\vskip 6 pt
\hrule
\vskip 3 pt
{\bf Answer: }
\begin{verbatim}
to basic2
  clear-all
  ask patches [if random 100 < 12
    [set pcolor yellow]]
  ask patches [if count neighbors with [pcolor = yellow] >= 3
    [set plabel "***"
      set pcolor red]
  ] 
end
\end{verbatim}
\vskip 3 pt
\hrule
\fi

\end{enumerate}

\section{Beyond the Basics}

\chapter{Working with Turtles}




%%%%%%%%%%%%%%%%%%%
\addcontentsline{toc}{chapter}{References}
%
\bibliographystyle{apalike} %{amsalpha} %plain}
\bibliography{../../sok,../../union}

\addcontentsline{toc}{chapter}{Index}
%\input 311s2015-teaching-notes-master.ind

\end{document}
%%%%%%%%%%%%%%%%%%%%%%%%%%%%%%%%%%%%%%%%%%
\item Write a function that conforms to the following template and documentation.

\begin{verbatim}
\end{verbatim}
When you run it, it should behave as follows.
\begin{verbatim}
\end{verbatim}

\ifnum\answers=1
\vskip 6 pt
\hrule
\vskip 3 pt
{\bf Answer: }
\begin{verbatim}

\end{verbatim}
\vskip 3 pt
\hrule
\fi
%%%%%%%%%%%%%%%%%%%%%%%%%%%%%%%%%%%%%%%%%%%%
