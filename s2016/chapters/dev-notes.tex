\chapter{Development Notes\label{ch:dev_notes}}



\section{2015-09-13}

So, I've begun working on the Teaching Notes for Thinking with Models (DOID 325) for spring 2016. The immediate goal is to get ready to teach the class in spring, especially with regard to in-class exercises, and notes for them. I plan to do this, focusing on the chapters (one per class) in Part I of the Notes. 

The intermediate goal is to design and work towards a new book, called {\it Thinking with Models}, that I can use for the class and that teaches, especially, solution pluralism and exploratory modeling. Puzzled a little about organization. One thing also is that I'm ambivalent about the longer term use of NetLogo. Why not just switch to Python? Well, there's a very nice NetLogo library and a lot of students chary of programming might find this ok. So, I'm thinking of doing follow on graphics and data analysis in Python, and maybe some models. Maybe something like this:

\begin{enumerate}
\item[Part I]  Programming NetLogo. 

Keyed initially to the Wilensky and Rand book. Certainly do this for spring 2016 (i.e., now!).

\item[Part II] Modeling examples.

Provide an early chapter surveying the field; kinds of models, drawing upon distinctions in chapter 1.

Already: Converse, Huff, Bueno de Mesquita. Coming: Solar PV Adoption, Ambidexterity Strategy Explorer, Probe and Adjust for Electricity Bidding. Probably: Models with Robin Clark.

What else? Could be Python models, even MATLAB models. Want to have models that afford deliberation and that connect with real decision making.

Here, I could use material from the BizAnalyticsBook with HC. Chapters here and in Part II.

\item[Part III] Exploratory modeling.

These chapters might be matched with chapters in Part II. Here I discuss exercising the models, solution pluralism, exploratory modeling, and I do lots of data analysis and graphics, all in Python.
\end{enumerate}
Well, this seems ok. Now to it! Note that there will be an accompanying document, {\it In-Class Exercises}.