% See my Daybook, pages 40--2, 7/18/07
\def\year{2016}
 \def\lastclass{**Wednesday, April 29, \year}
\def\noop#1{}
\def\place{\bf ***TBA***}
\def\canvas{\bf ***TBA***}
\def\mb{{\it DAMbook}}
\def\mlb{MATLAB}
\def\prf{{\it PRFbook}}
\documentclass[11pt]{article}
\usepackage[margin=1in]{geometry}                % See geometry.pdf to learn the layout options. There are lots.
\geometry{letterpaper}                   % ... or a4paper or a5paper or ... 
%\geometry{landscape}                % Activate for for rotated page geometry
%\usepackage[parfill]{parskip}    % Activate to begin paragraphs with an empty line rather than an indent
\usepackage{graphicx}
\usepackage{amssymb}
\usepackage{epstopdf}
\usepackage{hyperref}
\usepackage{natbib}
\usepackage{palatino} %{times}
\usepackage{color}
\DeclareGraphicsRule{.tif}{png}{.png}{`convert #1 `dirname #1`/`basename #1 .tif`.png}

\newcount\instructor

\instructor=1

\title{{\color{red} DRAFT:} Syllabus, Spring \year , for OIDD 325:  \\
Thinking with Models \\
		   3:00--4:20 M \&\ W, \place \\
		   Canvas: \canvas  %\url{https://wharton.instructure.com/courses/1220877}
		   }
\author{Professor Steven O. Kimbrough, Instructor  \\
Office hours: 565 JMHH, 9:30--11:00  and 1:30--3:00 Wednesdays, and by appointment}
%\date{April 28, 2014, revision 2}                                           % Activate to display a given date or no date

\begin{document}
\maketitle
%\section{}
%\subsection{}

%\url{http://www.youtube.com/watch?v=ZK3O402wf1c} Gilbert Strang on linear algebra.

\section{Class Description}

\section{Texts and Software}

%\cite{chou_kimbrough_cmot_2016}

\begin{itemize}
\item NetLogo. Free download from \url{http://ccl.northwestern.edu/netlogo/}.



\item  {\it NetLogo User Manual} (comes with NetLogo)

\item {\it An Introduction to Agent-Based Modeling}
\citep{wilensky_rand_2015}.

\item Other readings and handouts to include:
\begin{enumerate}
\item \citet{bankes_1993}
\item \citet{bankes_lempert_popper_2002}
\item \citet[chapters 1 and 2]{weisberg_2013}
\item See readings from fall 2014.
\end{enumerate}
\end{itemize}
\section{Grades}



\section{Class Schedule}

\begin{enumerate}
\item  Introduction and overview of the course. 

Reading (before class): \citep[chapter 0]{wilensky_rand_2015}, ``Why
Agent-Based Modeling.''

\ifnum\instructor=1
{\bf *** Instructor Notes: Overview of the class. SAIL philosophy. Initial discussion of what modeling is
  about and the three distinctions: parametric versus strategic,
  insight versus decision, and description versus exploration. We are
  largely:
parametric,  insight, exploration. NetLogo: about,
examples. Installing NetLogo. Note that we will focus on programming
NetLogo at the start, then move on to other matters to do with
modeling once you have this under your belts.
What next. Should we do the initial exercises? Probably reserve that
for next time. ***}
\fi

\item Getting started with ABM.

Reading (before class): \citep[chapter 1]{wilensky_rand_2015}, ``Why
Agent-Based Modeling'' and from the {\it NetLogo User Manual}
\begin{itemize}
\item Learning NetLogo
\begin{itemize}
\item   Tutorial \#1: Models
 %\item  Tutorial \#2: Commands
% \item  Tutorial \#3: Procedures 
\end{itemize}
\end{itemize}

\ifnum\instructor=1
{\bf *** Instructor Notes: We want to make this a prototypical class,
  I think. Begin with Q\&A and perhaps a 20 minute lecture/overview of
the material for the day. Then onto SAILing. Perhaps both some NetLogo
exercises (the first, very simple ones) and also some small group
exercises, to be written down and handed in? Roughly: 20 minutes
lecture, 30 minutes NetLogo exercises, 30 minutes small group
exercises? ***}
\fi
\item Simple ABMs: Life.

Reading (before class): \citep[chapter 2, pages 45--68]{wilensky_rand_2015},
``Creating Simple Agent-Based Models'' and from the {\it NetLogo User Manual}
\begin{itemize}
\item Learning NetLogo
\begin{itemize}
%\item   Tutorial \#1: Models
 \item  Tutorial \#2: Commands
 \item  Tutorial \#3: Procedures 
\end{itemize}
\end{itemize}

\item Simple ABMs: Heros and Cowards.

Reading (before class): \citep[chapter 2, pages 68--87]{wilensky_rand_2015},
``Creating Simple Agent-Based Models''.

\item Simple ABMs: Simple Economy.

Reading (before class): \citep[chapter 2, pages 87--99]{wilensky_rand_2015},
``Creating Simple Agent-Based Models''.

\ifnum\instructor=1
{\bf *** Instructor Notes: Up through this class things are pretty
  much slow and easy. The in-class exercises will be key to
  success. They should include both programming and small group exercises. ***}
\fi

\item NetLogo as a Conventional Programming Lanuage.

Reading (before class): The Info tab of the Conventional Programming 1
NetLogo model, found on the Modeling Commons
(\url{modelingcommons.org}, search ``kimbrough'').

\ifnum\instructor=1
{\bf *** Instructor Notes: Have in mind a longer lecture, maybe 30--40
  minutes, with examples, then programming exercises. ***}
\fi

\item Exploring and Extending Agent-Based Models, 1.

Reading (before class): \citep[chapter 3, pages 101--128]{wilensky_rand_2015},
``Exploring and Extending Agent-Based Models''.

\ifnum\instructor=1
{\bf *** Instructor Notes: Cover the Fire model and the DLA model
  briefly in class, along with key programming material, then set them
  to work on (a) programming exercises and (b) small group exercises. ***}
\fi

\item Exploring and Extending Agent-Based Models, 2.

Reading (before class): \citep[chapter 3, pages 128--153]{wilensky_rand_2015},
``Exploring and Extending Agent-Based Models''.

\ifnum\instructor=1
{\bf *** Instructor Notes: Cover the Segregation model and the El Farol model
  briefly in class, along with key programming material, then set them
  to work on (a) programming exercises and (b) small group exercises. ***}
\fi

\item Creating Agent-Based Models, 1.

Reading (before class): \citep[chapter 4, pages 157--189]{wilensky_rand_2015},
``Creating Agent-Based Models''.

\ifnum\instructor=1
{\bf *** Instructor Notes: Cover the readings
  briefly in class, along with key programming material, then set them
  to work on (a) programming exercises and (b) small group exercises. ***}
\fi

\item Creating Agent-Based Models, 2.

Reading (before class): \citep[chapter 4, pages 189--197]{wilensky_rand_2015},
``Creating Agent-Based Models''.

\ifnum\instructor=1
{\bf *** Instructor Notes: Cover the readings
  briefly in class, along with key programming material. Also lecture
  and/or have a reading on multiple runs/solution pluralism.  Then set them
  to work on (a) programming exercises and (b) small group exercises. ***}
\fi

\item Chapter 5, part 1

\item Chapter 5, part 2

\item Chapter 6, part 1

\item Chapter 6, part 2

\item Chapter 7, part 1

\item Chapter 7, part 2

\item And then \ldots ??

Suggestions:
\item What makes one model preferable to another? Introduction to various epistemic standards.
\item Is simulation different from analytical analysis (``same old stew?''), with examples from epidemology and game theory.
\item How to test a model? Simple regressions and experiments?
\item Metaheuristics, genetic algorithms and the blind watchmaker

\end{enumerate}
\vskip 12 pt
{\bf Small group assignment hand-ins due: 5 p.m.\ Sunday, May 1, \year.} \marginpar{\bf Small group hand-ins.}


%%%%%%%%%



%
\newpage

\section{Calendar, Spring \year}

Reading days: April 28--9, \year . Final examinations: May 2--10, \year .


\begin{table}[h]
\centering
\begin{tabular}{|c||c|c|c|} \hline
   & 0 & 1 & 2  \\ \hline\hline
  0 & ---  & W: \year -02-17 & W: \year -03-30 \\
1 & W: \year -01-13 & M: \year -02-22 & M: \year -04-04 \\
2 & W: \year -01-20 & W: \year -02-24 & W: \year -04-06 \\
3 & M: \year -01-25 & M: \year -02-29 & M: \year -04-11 \\
4 & W: \year -01-27 & W: \year -03-02 & W: \year -04-13 \\
5 & M: \year -02-01 & M: \year -03-14 &  M: \year -04-18 \\
6 & W: \year -02-03 & W: \year -03-16 & W: \year -04-20 \\
7 & M: \year -02-08 & M: \year -03-21 & M: \year -04-25 \\
8 & W: \year -02-10 & W: \year -03-23  &  W: \year -04-27\\
9 & M: \year -02-15 & M: \year -03-28  & --- \\
\hline
\end{tabular}
\caption{Class number ::  date correlation, for Monday (M) and Wednesday (W) classes, spring \year .}
\end{table}

\bibliographystyle{apalike} %{amsalpha}
\bibliography{../../sok,../../union}
\vfill
\noindent\verb+$Id: syllabus-doid325-s2016.tex 4938 2015-09-08 21:17:53Z sok $+

\end{document}  
